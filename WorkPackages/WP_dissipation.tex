\begin{workpackage}[id=WPdissipation,wphases=0-48,
short=Dissipation,
title=Harnessing dissipation,
lead=TUE,
TUERM=42,
KULRM=6]

\newrefsection

\begin{wpobjectives}
\begin{compactitem}
\item Mathematical design of information-processing properties of nonequilibrium probes.
\item To identify the constructive and controllable elements in the time-symmetric part of transition rates that are influenced under nonequilibrium conditions.
\item Provide the mathematical basis for a prototype frenometer which would yield operational and experimentally accessible aspects of dynamical activity in the phenomenology of nonequilibrium thermodynamics.
\end{compactitem}
\end{wpobjectives}

\begin{wpdescription}
  % Overall description; typically 10 lines to half a page
  % as appropriate, depending on the variety and number of tasks
Recently it has become recognized that living cells make use of energy dissipation to reduce noise, increase sensitivity, and generally improve their information-processing capabilities. This is a non-trivial feat, since dissipation typically destroys correlations and eliminates information. Man-made microprobes and bioprobes could greatly benefit from this same possibility, provided we can understand and tune it appropriately. In addition, dissipation is energetically wasteful, which constitutes a problem for autonomous sensors. Careful control is necessary. 

In this project we therefore investigate the mechanisms and principles that underlie the beneficial sides of the usually harmful dissipation. The focus will be on extracting from examples the general guiding principles, allowing us to leverage this understanding for new applications.  In particular we will connect with the work of the Leuven-Padova-Prague nodes to identify the role of dynamical activity in response and fluctuation behavior.
Their constructions of frenometers (new word for the operational control and measurement of time-symmetric activity in nonequilibria) will be tested and confronted with specific situations of nonequilibrium thermodynamics.
Exchanges with Leuven will happen at a rate of visits of one day per month.

The work in this package will be possible by a combination of ingredients. We will leverage the link we recently discovered~\cite{AdamsDirrPeletierZimmer13} between large deviations and gradient flows to connect the macroscopic dissipation with microscopic features of the underlying dynamics. In addition, recent developments in chemical network theory give a tighter connection between stochastic dynamics and dissipation (e.g.~\cite{PolettiniWachtelEsposito15}). The work will also greatly benefit from the expertise of Leuven on nonequilibrium statistical physics and of Stuttgart on coarse-graining. 
Finally, the group in Eindhoven has an ongoing collaboration with an experimental group at the Institute for Complex Molecular Systems in Eindhoven, headed by Tom de Greef (see e.g.~\cite{RoekelMeijerMasroorGarzaEstevez-TorresRondelezZagarisPeletierHilbersGreef14}). This group focuses on bottom-up synthetic biology, and can provide essential chemical and biological insight. 

%
% test comments
%

% Sykes, A Guidebook to Mechanism in Organic Chemistry
\end{wpdescription}

% Please see UserInterfaces.tex for now as an example

\begin{tasklist}
  % 3-5 tasks

  % The description of each task can be 5 to 15 lines depending on the
  % complexity and amount of details deemed necessary, and involve and
  % refer to 1-3 deliverables.

  \begin{task}[title= Specification of dissipation in bistable switches,id=diss-t1,lead=TUE,wphases=0-12!1.0]
  We first focus on a single information-processing step, the bistable switch (a flip-flop), which is a core element of silicon-based computing, and which also exists in chemical form. For single examples the relation between dissipation and performance has been studied (e.g.~\cite{LanSartoriNeumannSourjikTu12,LanTu13,MehtaSchwab12,CaoWangOuyangTu15}). In this task we generalize from examples to a general principle that explains the underlying reasons.
  \end{task}

  \begin{task}[title=Operational control of constructive role of noise,id=diss-t2,lead=TUE,wphases=12-30!1.0,partners={KUL}]
  We next include time. An auto-nulling amplifier is a simple input-output information-processing unit, that temporarily amplifies differences in the input, and resets itself after some time. In earlier work (publication in preparation) we showed that such an object can be chemically constructed completely without dissipation, but only in a zero-noise context. We now include noise, and study how dissipation can be used to reduce the noise and improve the gain.  \end{task}
  
  \begin{task}[title=Solving toy-models for harnessing dissipation in operational units,id=diss-t3,lead=TUE,wphases=30-48!1.0]
A chemical oscillator is an intrinsically nonequilibrium object, that processes information by reacting in period and phase to external input. Dissipation is essential for the oscillator itself, and in this task we first study the role and limits provided by dissipation on the oscillator behaviour. We then continue with the effect of external inputs, and the role of dissipation in controlling this effect.    

\end{task}
  

\end{tasklist}

\printbibliography[heading=proposal-bib,env=proposal-env]

\eucommentary{ Deliverable numbers in order of delivery
  dates. Please use the numbering convention ``WP number''.``number of
  deliverable within that WP''.  For example, deliverable 4.2 would
  be the second deliverable from work package 4.
%
  Type:
  Use one of the following codes:
  R: Document, report (excluding the periodic and final reports)
  DEM: Demonstrator, pilot, prototype, plan designs
  DEC: Websites, patents filing, press \& media actions, videos, etc.
  OTHER: Software, technical diagram, etc.
  Dissemination level:
  Use one of the following codes:
  PU = Public, fully open, e.g. web
  CO = Confidential, restricted under conditions set out in Model Grant Agreement
  CI = Classified, information as referred to in Commission Decision 2001/844/EC.
  Delivery date
  Measured in months from the project start date (month 1)
}
\begin{wpdelivs}
  \begin{wpdeliv}[due=24,id=wp-diss-1,dissem=PU,nature=R,lead=TUE]
    {Mathematical road-map and characterization for the role of dissipation in the bistable switch.}
  \end{wpdeliv}
  \begin{wpdeliv}[due=36,id=wp-diss-2,dissem=PU,nature=DEM,lead=TUE]
    {DEM: Design of noise-control  in an auto-nulling amplifier.}
  \end{wpdeliv}
  \begin{wpdeliv}[due=48,id=wp-diss-3,dissem=PU,nature=R,lead=TUE]
    {Complete solution of the role of dissipation in nonequilibrium oscillator models.}
  \end{wpdeliv}
\end{wpdelivs}
\end{workpackage}

%%% Local Variables:
%%% mode: latex
%%% TeX-master: "../proposal"
%%% End:
