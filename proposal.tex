\providecommand{\classoptions}{keys}
\documentclass[noworkareas,deliverables,\classoptions]{euproposal}       % for writing
%\documentclass[submit,noworkareas,deliverables]{euproposal}        % for submission
%\documentclass[submit,public,noworkareas,deliverables]{euproposal} % for public version

\usepackage[utf8]{inputenc}

\usepackage{float}  % used to suppress floating of tables in Resources section.
\usetikzlibrary{calc,fit,positioning,shapes,arrows,snakes}

\addbibresource{bibliography.bib}
\input{WApersons} % Some sections of the included files depend on this.
\usepackage{comments}
\usepackage{framed}

\defbibenvironment{proposal-env}
{\inparaenum%
\renewcommand*\labelenumi{[\theenumi]}}
{\endinparaenum}
{ \item}

\setdefaultleftmargin{1em}{0.5em}{0.5em}{0.5em}{0.5em}{0.5em}

\defbibheading{proposal-bib}[References:]{\noindent {\bf #1}}

\begin{document}

\begin{proposal}[
  % These PM numbers (person months) are for the coordinator to help planning
  % Participants should not change these, but add PM numbers in the CVS in
  % the site descriptions at CVs/*.tex
  site=KUL,
  site=TUE,
  site=ULEI,
  site=UNIPD,
  site=FZU,
  site=USTUTT,
  botupPM, % we want to work via bottom up PM distribution,
  coordinator=cmaes,
  coordinatorsite=KUL,
  acronym={PROMANE},
  acrolong={PROMANE},
  title=Probing Macroscopic Nonequilibria,
  callname=FET-OPEN Research and Innovation actions 2014-2015,
  callid=H2020-FETOPEN-2014-2015-RIA,
  keywords={nonequilibrium thermodynamics, new materials, transport and control theory},
  instrument=Horizon 2020 - Future and Emerging Technologies,
  months=48,
  compactht]
\newcommand{\TheProject}{\pn}% \pn is defined automatically

\begin{abstract}

\TheProject will develop a control methodology within nonequilibrium statistical
thermodynamics, promoting irreversible thermodynamics and dynamical systems to a unified tool for
industrial
applications which includes the recent progress in fluctuation and response
theory.
New technologies will benefit from the manipulations of and the control via macroscopic
nonequilibria. These are nonlinear systems that are are subject to nonconservative forces or
driven from contacts with multiple reservoirs that impose conflicting thermodynamic
behavior, or for which the constituents are active, {\it i.e.}, objects that could be
self-propelling. Our main hypothesis is that macroscopic nonequilibria can be used for
steering immersed subsystems or probes, thus developing a control theory using nonequilibria
together with the possibility of creating new macroscopic phases.
%
The expected deliverables include enhanced stabilization of metastable phases and the
emergence of new response and transport behavior, crucial items in developing systematically
interesting devices and material properties.
For technological avenues, our work will reveal important insights, experimental and
computational methods for example by providing far-from-equilibrium relaxation times of
viscoelastic media, for targeted catalysis, for clustering control or for viscosity
reduction in industrial processing. Other possible applications involve noninvasive
surgeries and drug delivery in medicine, or the design of smart materials with built-in
actuation mechanisms.
%
Previous efforts focused on the destructive and information-erasure roles of dissipation.
Time has come to comprehend and exploit the
constructive role and possibilities of nonequilibrium conditions. Complementary to entropy
production, changes in kinetics and dynamical activity make statistical forces nongradient
and nonadditive, allowing unseen and novel behavior. Controlling it opens a new paradigm and major
opportunities for future technology.

\end{abstract}

\newpage

\ifsubmit\else\setcounter{tocdepth}{4}\fi
\tableofcontents


% ---------------------------------------------------------------------------
%  Section 1: Excellence
% ---------------------------------------------------------------------------

\section{S\&T Excellence}
\subsection{Targeted breakthrough, Long term vision and Objectives}\label{sec:objectives}

\eucommentary{
  \begin{compactitem}
  \item Describe the targeted scientific breakthrough of the project.
  \item Describe how the targeted breakthrough of the project contributes to a
    long-term vision for new technologies.
  \item Describe the specific objectives for the project, which should be clear,
    measurable, realistic and achievable within the duration of the project.
  \end{compactitem}
}

\paragraph{Scientific and technological goals}

The project is aimed at exploiting and controlling the motion of probes and devices in and
through contact with nonequilibrium media. ``Nonequilibrium media'' refers to an environment
(solvent, bulk of a material, etc) that is active or that is driven away from its
thermodynamic equilibrium.
%
The {\bf scientific
  breakthrough} involves an operational formulation of nonequilibrium statistical mechanics.
That goal implies the transition to a third major stage of {\bf fundamental nonequilibrium
  research}. Most often in the past, the focus has been either on studying the approach to equilibrium or on
the derivation and description of dissipative transport of mass, energy and momentum between
various equilibrium reservoirs.  By deriving and using the precise relation between
systematic forces, friction, and noise on probes, we open a new and realistic class of
dynamical systems for which the control lies in the nonequilibrium environment.
Complementarily, the understanding of the fluctuation and response behaviour of nonequilibrium
media allows us to extend the standard irreversible thermodynamic theory as it was conceived in
1930--1970.

Both, new control of emerging dynamical systems as probes connected to nonequilibria and the manipulation of kinetic to thermodynamic behaviour of media are essential
ingredients in future technology and industrial applications.  To cite a few general
far-reaching objectives: the controlled motion or transport of material through living
tissue or in interaction with life processes requires fundamental departures from
traditional transport theory; the stabilization of coherence as needed for future quantum
technology cannot be obtained within thermal environments but could be enhanced via
nonequilibrium contacts; mobilities and conductivities close-to-equilibrium are subject to
the fluctuation-dissipation relation but nonequilibrium driving can produce unseen transport
and conductivity behavior for the realization of new materials. 


 As is central to the goals
of the FET-open initiative, theoretical and experimental foundational work will lead to
create abilities and conditions under which technological innovation becomes possible.  For
example, serious conceptual problems need to be solved before active-particle suspension
will become a common basis for enhanced targeted catalysis or for fluid mixing and viscosity
reduction in industrial processing, or for noninvasive surgeries and drug delivery in
medicine, or for the design of smart materials with built-in actuation mechanisms. These
open conceptual issues are addressed via several pathways in the theoretical parts of the
proposal.  Leaping into macroscopic nonequilibria moreover implies that classical
measurement techniques, developed for equilibrium systems, will generally fail to work as
usual and will have to be replaced by new designs. Brownian thermometry provides an example
that is relevant to a variety of applications \cite{kroy:2014}, and which will be
generalized to conditions far from equilibrium, in the project.  On the practical and
applied side, radically new manipulation techniques are required to fully exploit the
innovative potential provided by active particles. As pertinent examples, we mention the
thermophoretic trap and the photon-nudging method pioneered by the experimental group in
Leipzig \cite{Qian2013,Braun:NanoLetters:2015}.  Both techniques steer particles (passive
and active particles in the trap and hot active particles in the nudging technique,
respectively) by exploiting far-from equilibrium conditions in the solvent. Notably, they
work without imposing external forces, via directly addressing the nonequilibrium solvent
conditions and thereby the (self-)thermophoretic propulsion (or ''swimming'') of the
particles. Their usefulness is enhanced by employing Maxwell-demon type control
strategies. Prototypes of these genuinely nonequilibrium techniques exist and will be
employed and refined (e.g. by optimizing dedicated control strategies) within the project.
One should realize that the assumption of a thermally equilibrated bath is not fulfilled in
most industrially (oil, polymer melts, dense colloidal or micellar suspensions) or
biologically relevant (blood, mucus, DNA-solutions) liquids which display visco-elastic
properties, i.e. elastic behaviour like a solid and viscous behavior like a fluid. As a
consequence, such fluids exhibit large relaxation times (up to seconds), and can thus easily
be driven out of thermal equilibrium by a forced colloidal probe. Within this proposal, we
want to study how the local perturbation of visco-elastic systems affects the dynamics and
effective interactions of colloidal particles in such media with particular view on
e.g. memory effects, many body interactions and local non-linear rheological properties.
Similarly, time-dependent phenomena at very long and even at very fast time scales, such as
in glasses or electronic devices, elude the usual equilibrium thermodynamic
approach.  Such studies will in general reveal the properties of nonequilibrium baths which
are of manifold importance.

More generally and as part of the {\bf long term vision} we believe that a systematic
understanding of nonequilibrium response and fluctuation behavior, beyond the traditional
linear response regime, will uncover new parameters to control technological processes, to
steer motion, and to explore and stabilize new interesting material properties.

\paragraph{Objectives}

can be divided into two major classes:\newline
\begin{inparaenum}[A.]
\item exploiting the control and manipulation of probes and devices in contact with nonequilibrium media.  That means to understand the emerging dynamical system immersed in nonequilibria.  It leads to better controlled transport in environments as turbulent atmosphere, active media or nonlinear fluids.  Besides transport, we expect also stabilization and emergence of unseen device properties such as long time coherence and classical dia- and paramagnetic phases, to name a few;\newline
\item manipulating nonequilibrium media for their response behavior.  The addition of kinetic and nonequilibrium parameters in material design can lead to absolutely new effects in response or susceptibility. Negative compressibilities, thermal or variable conductivities introduce new possibilities and applications for flexible and smart materials.  We will provide a new computational framework for {\it ab initio} calculations of nonequilibirum phase diagrams, opening a new important tool in material and device production.
\end{inparaenum}


Besides the theoretical and experimental underpinning of the recent insights and progress in
nonequilibrium science, replacing much of the current trial and error methods as mentioned
above, part of the general objective includes to realize numerical and algorithmic
methods for efficient simulation of nonequilibrium environments.
We will provide modelling and simulation techniques for the simulation of probes and
devices in contact with nonequilibria.  These are high level molecular dynamics codes for
simulating and calculating the emergent behavior of contacts and probes.  These techniques
will be made available through all  work packages. It is an important objective
to make that tool also available to industrial partners, challenged by the complexity of
dealing with nonequilibria.\newline
%
% Remark from Mark Peletier about the preceding line:
% Isn't it true that this proposal has no industrial partners?
%
We will offer a scientific basis and new avenues for the realization of new properties of
matter (in response and transport), and the enhanced stability of thermodynamic phases or
behaviour.  See further also in Work Package 2 for the conception of
meta-materials. Specifically we have good understanding for workable implementations of
negative compressibilities.\newline
For the control of colloidal motion in nonequilibrium baths and in nonlinear media we launch (Work Package 3) theoretical and experimental studies of probes in visco-elastic
solvents.  Most fluids found in industry and biology are visco-elastic and/or out of
equilibrium.  Control of shear, the development of new separation techniques based on
dynamical activities and steering of clustering properties are achievable in the current
project.\newline
Many challenges are kinetic rather
than thermodynamic, especially for processes and manipulations in nonequilibrium
environments or for time-dependent parameters (Work package 4).  Irreversible thermodynamics cannot deal with
the overwhelming nature of kinetic control, but nonequilibrium statistical insights point to
the control of dynamical activity; what some of us have called the frenetic contribution.
One objective is to make a frenometer for operational control and measurement of dynamical
activity.  That is essential for the control and steering of dynamics out-of equilibrium.
For example, polymer physics outside equilibrium will supply a necessary and complementary
component to the vast domain of polymer research with thermodynamic control.\newline
We will develop the theory of statistical forces on probes in contact with macroscopic
nonequilibria, essential for controlling motion in turbulent media (atmosphere dynamics) or
under biological flow (blood streams).  Work Packages 5 and 6 will work with experimental
and observational predictions of new emergent behavior for probes in nonequilibria, as
resulting from the essentially nongradient, nonadditive and nonlocal features of these
nonequilibrium statistical forces.


These objectives will find further realizations in our existing contacts with colloidal and
polymer scientists, micro-electronic and microbio-engineers and atmosphere researchers. Related topics are for
example found in the physics of polymer folding and for the phase diagram of bio-polymers, the
mechanical properties of the cytoskeleton and fluctuations of bio-membranes, and the
statistical analysis and development of climate and weather models. {\bf The overarching goal
remains to make available the science of nonequilibria to technological and societal
developments.}



\subsection{Relation to the work programme}\label{sec:relation-wp}

\begin{compactdesc}
\item[Long-term vision] \TheProject enables a change of paradigm across many technological
fields on the basis of nonequilibrium thermodynamics. Current technologies rely on an
understanding of thermodynamics that dates from half a century ago and we will deliver the
set of tools to scientists and to the industry that is missing.
\item[Breakthrough S\&T target] The developments within \TheProject rely directly on
technological applications, either via relevant physical models or via the experimental
partners (\site{USTUTT} and \site{ULEI}). Advanced or novel nonequilibrium theories are
currently not used in the industry and we will provide the building blocks to enable them
for the development of new material (\WPref{WPcompress}), ultra fast processes (see XXX)
and control strategies (\WPref{WPdissipation}) and viscoelastic media (\WPref{WPbrown}).
\item[Novelty] The changes that we are proposing do not aim at a minor modification of the
existing processes but at replacing them with a better understanding and, more specifically,
new strategies for diagnosis and control.
\item[Foundational] Our work will have a decisive impact on other fields in which the
control of devices lacks a comprehensive thermodynamical understanding: medicine and
pharmacology (control of nanoparticles in living systems and targeted delivery mechanisms),
engineering (fast industrial processes, see XXX).
\item[High risk] Scientific research carries risks by its very nature, as we are pushing the
frontiers of knowledge. It may be the case that our ideas stumble on mathematical or
experimental challenges that would require efforts beyond the 4 yeards of the
project. Still, the combined expertise in \TheProject offers strong chances of success and
even before completion many exciting outputs are foreseen (see XXX).
\item[Interdisciplinarity] \TheProject consists in two experimental physics groups
(\site{ULEI}, \site{USTUTT}), one mathematical group (\site{TUE}), two theoretical physics
groups (\site{KUL}, \site{FZU}) and one group specialized in simulation methods
(\site{UNIPD}). This combination allows a consistent organization of the work across work
packages to deliver results ranging from the conceptual to the proof of principle.
\end{compactdesc}

\subsection{Novelty, level of ambition and foundational character}\label{sec:progress}

\eucommentary{
  \begin{compactitem}
  \item Describe the advance your proposal would provide beyond the
    state-of-the-art, and to what extent the proposed work is ambitious, novel
    and of a foundational nature. Your answer could refer to the ground-breaking
    nature of the objectives, concepts involved, issues and problems to be
    addressed, and approaches and methods to be used.
  \end{compactitem}
}

The first stage of modern thermodynamics consisted in defining equilibrium systems, a field
that gave rise to equilibrium statistical mechanics. The current knowledge of nonequilibrium
physics represents, as we have outlined in Sec.~\ref{sec:objectives}, the {\em second stage}
of the domain: the study of transport and response theory close to
equilibrium.

\paragraph{Non dissipative nonequilibrium physics}
Since about two decades, nonequilibrium studies have revisited steady regimes further away
from equilibrium, and a fluctuation-response theory is starting to emerge.
The recently developed idea of dynamical activity starts a new line of conceptual
understanding that is missing from today's nonequilibrium thermodynamics. The focus on
entropy production has brought many insights in the last 150 years (in physics and other
fields as well, such as information theory) but is however restrictive.  We emphasize the role of excess thermodynamic quantities in relaxation to {\it non}equilibria, and we find that an important role is also being played by time-symmetric fluctuations.
Volatilities or time-symmetric traffic, frenesy and other names have been given to this dynamical activity which is so important  for a valid fluctuation-response theory away from equilibrium.
 Our current
understanding of time-symmetric quantities remains rather formal and most results concern
specific systems (often ``toy models'') , be it of stochastic or highly chaotic nature.
%
\TheProject chooses specific questions, some mathematically very challenging and some very
pragmatic, in fluctuation-response theory to open the meaning of dynamical activity, and to
make it operational.
%
Explicitly, we will focus on consequences that can be observed and measured, and on the
experimental control that this theory brings.

\paragraph{Probing nonequilibrium}

We will set up a fluctuation and response theory for nonequilibrium media from their
influence on probe dynamics and implement the control and monitoring of a system's behavior
from its contact with a nonequilibrium medium.
%
The point is to explore the
possible very new behaviour and properties of systems in contact with nonequilibrium
media.
%
A first question one can ask about a probe in active contact with a nonequilibrium
sea is about the induced systematic statistical forces of the sea on the probe.
%
One starts from the more microscopic mechanical force and, assuming that the probe changes
its state (e.g. position) on a much slower timescale than the medium, one integrates that
mechanical force over the degrees of freedom of the stationary medium.
%
In equilibrium (that
is, when the probe is in contact with an equilibrium reservoir) the resulting (statistical)
forces are of gradient type.  The standard thermodynamic potentials are (indeed) the
potentials from which the force can be derived as in Newtonian mechanics. The result there
is that it allows us to draw free energy landscapes to understand the changes in the system,
as if it concerned a conservative mechanical system for which we give the potential energy.
A special interesting example are the entropic forces, which work by the power of large
numbers; then, the free energy has negligible energetic or enthalpic contributions.  These
are known to be important in elasticity and polymer physics. Moreover, always for
equilibrium, the statistical forces are additive for example in the sense that bulk
contributions can easily be separated from boundary contributions.  The latter is crucial
for thermodynamic behaviour.  For example, the very possibility of thermodynamic behaviour,
the presence of an equation of state etc depend on that distinction.  All of that need not
and in general, will not, be true for probes (systems, walls, collective coordinates) in
contact with (genuine) nonequilibrium media or reservoirs. That has possibly interesting
consequences in the manipulation of such forces, adding oscillatory components, going from
attractive to repulsive and obtaining nonvanishing resulting forces that otherwise, in
equilibrium, would be zero from mutually canceling contributions.
%
It also means that the stationary positions of the probe (quasi-static) would be at
different locations compared with equilibrium, allowing possibly increased stability of
phases that would otherwise (in equilibrium again) be unstable.  The paradigmatic example is
here the Kapitza oscillator (pendulum) which remains stable upright when being shaken.  That
example is however likely to be systematized also outside the theory of dynamical systems,
reaching then in this project the stability of macroscopic behaviour of the probe
(collective coordinate) in for equilibrium very unlikely phases and values for order
parameters.  Needless to add here that this may have dramatic consequences on the phase
diagram for systems in contact with nonequilibrium media, not only breaking the Gibbs phase
rule but also introducing new phases of matter.  That is certainly one of the most exciting
possibilities of explorations in the present project. As another theme of stability we want
to understand some mechanism of homeostasis. In our set-up we treat spatially extended
systems with time-dependent boundary driving. We will see under what conditions the bulk of
the system reaches a steady (time-independent) regime. A related consequence of contact with
nonequilibrium reservoirs is the possibility of population inversion in the system, which
will mean that the effective temperature of the probe could be much larger for certain
purposes.  There will for sure not be the usual Einstein or second fluctuation-dissipation
relation between noise and friction on the probe which also entails that the friction
coefficient can show further nontrivial behaviour.  


\subsection{Research methods}\label{sec:methods}
\eucommentary{
  \begin{compactitem}
  \item Describe the overall research approach, the methodology and explain its
    relevance to the objectives.
  \item Where relevant, describe how sex and/or gender analysis is taken into
    account in the project's content.
  \end{compactitem}
}

The overall research approach is that of modern physics, with its traditional ways of open
exchange of ideas and results.  As generic and general as that may sound, the history of the
last 200 years proves that these make the best and most direct road to innovative technology
research.  The fundamentally connected components in this research have mathematical,
conceptual, computational and experimental sides.
%
We certainly must work to keep up and even improve the exchanges between these efforts and
to bring about and foster multiple exchanges of ideas and questions.  The way from theory to
experiment is also the method to reach more applied science and eventually to influence
industrial and economic research centers and technology.  About all of the project members
have been asked before in consulting with industrial projects (e.g. water transfer in porous
media, new challenges for construction physics, crack evolution and creep in glassy
materials etc) but the present project will allow systematic developments in contacting and
helping industrial players.


1. Mathematical methods: Here we mostly follow stochastic calculus and the specialized
probabilistic techniques of dealing with spatially extended stochastic dynamics of
interacting particle systems. Limit theorems and large deviation theory are central tools.
The Eindhoven node is in the Applied Mathematics department, and Leuven and Prague members
are part of the research group in mathematical physics.

2. Theoretical framework: Much emphasis remains on path-integral techniques in dynamical
ensembles.  Such a Lagrangian statistical mechanics takes up the entropy fluxes and the
changes in dynamical activity to build the action functional for dynamical ensembles.  The
nonequilibrium techniques will be essentially non-perturbative in the driving, and keeping a
distance from the Keldysh formalism. Strong theoretical efforts are available in the
Stuttgart Max Planck Institute for Intelligent Systems, and in the Theoretical Physics
Institutes of Prague, Leuven, Padova and Leipzig.

3. Computational support: For various model systems that are not covered by analytic results, we use numerical methods and large scale simulation.  Main experts are
here found in Leuven and in Padova, with various schooling and outreach initiatives to
promote the computational component.  Molecular dynamics simulations will play a very big role, also because they reach more realistic scenario's that can be implemented and further verified in experimental work and tests.

4. Experimental work: Here we mostly undertake experiments on active materials and particles
in the appropriate soft condensed and liquid matter labs.  Other questions relate more to
biophysics and polymer behavior. The experimental component is present in the Stuttgart and
Leipzig labs.  The experimental side is much based on methods in fluid mechanics and optical control


5. Research\&Development.  Our groups are in close contact with research institutes having industrial anc commercial partners.
As an example, imec in Leuven is very close and mutually involved withing the physics department in Leuven; the PI has students and collaborations there.
A market survey and business plan is under construction that will possibly include the implementation of nonequilibrum techniques in quantum machines.


Overall, this project will also deliver methodology and development tools, such as
\begin{inparaenum}[A.]
	\item the making available of strong molecular dynamics simulation codes for the behavior of
	probes in nonequilibrium media and for the dynamics of active particles;
	\item the development of operational tools for measuring and quantifying kinetic parameters
	as used in the theory of dynamical ensembles, including the measurement of excess
	thermodynamic quantities and dynamical activities;
	\item providing a toolkit for {\it ab initio} calculations for macroscopic nonequilibrium
	behavior, possibly allowing new stable phases, including a quantitative description of
	(also) not-dissipative relaxation to stationary and steady conditions.
\end{inparaenum}


\subsection{Interdisciplinary nature}\label{sec:interdisc}

\eucommentary{
  \begin{compactitem}
  \item Describe the research disciplines involved and the added value of the inter-disciplinarity.
  \end{compactitem}
}

Statistical and mathematical physics are by their nature inter-disciplinary.  Not
only is there very often a large intersection with mathematics, statistics and computer
science, but also the subjects are often very diverse and reach out in many various
directions.  The present project, by its foundational character, plays then also on various levels and domains of development. We emphasize the nonequilibrium nature of
phenomena and processes making contact with condensed matter physics, with
fluid mechanics and with biological and medical/engineering sciences.  That is quite obvious
as nonequilibria are ubiquitous.  Discovering common grounds of study and constructing for
them a nonequilibrium statistical mechanics is exactly  at the heart of the present programme.

Research disciplines that are involved are very diverse. The most amazing examples of
nonequilibrium are probably found in life processes, or perhaps in the origin of life
itself.  There we see an open system full of transport processes, with little engines, pumps
and cycles and the emergence of order on diverse spatio-temporal scales out of molecular
complexity. In biology, molecular motors give
mesoscopic realizations of chemical engines and of transport on the molecular scales.  The
cell itself, how it moves and how its membrane fluctuates on $\mu m$-scales, brings
nonequilibrium statistical mechanics to life.  Chemical reactions are traditionally also
sources of nonequilibrium changes.  The variety of phenomena where nonequilibrium considerations are essential is
however much larger.  We find them at the smallest sizes of nanotechnology. Fluctuations at
the smallest dimensions produce sources of noise and matter, from which our world is finally
made.  Turbulent flow and nonlinear systems give macroscopic realizations of nonequilibrium
effects.  Climatology and ecology ask questions about the atmosphere and ocean dynamics and
about food web chains as nonequilibrium systems.  





%%% Local Variables:
%%% mode: latex
%%% TeX-master: "proposal"
%%% End:


% ---------------------------------------------------------------------------
%  Section 2: Impact
% ---------------------------------------------------------------------------

\section{Impact}
\subsection{Expected impacts}

\eucommentary{
  Please be specific, and provide only information that applies to the proposal
  and its objectives. Wherever possible, use quantified indicators and targets.
%
  \begin{compactitem}
  \item Describe how your project will contribute to the expected impacts set out
    in the work programme under the relevant topic.
  \item Describe the importance of the technological outcome with regards to its
    transformational impact on science, technology and/or society.
  \item Describe the empowerment of new and high-potential actors towards future
    technological leadership.
  \end{compactitem}
}

Statistical mechanics by itself had an enormous impact on technological and economic
developments in the last century. Its framework not only inspired the logical framework
of modeling and analysing microscopics to reach mesoscopic and macroscopic scales, but it
has very specifically enabled {\it ab initio} schemes and algorithms to reach and predict
macroscopic behavior.
%
The theoretical foundations of solid state physics and quantum field theory that have reshaped the
modern world and brought major changes in daily life invariably are consequences of the
fluctuation theories pioneered in statistical mechanics. Yet most, if not all, of that is
restricted to equilibrium statistical mechanics with its powerful and general Gibbs
machinery.
%
While that formalism and the consequent algorithms and implementations remain a superb tool
in technological research and innovation, the present project turns to nonequilibrium theory
and phenomenology. There we reach even much richer grounds, also connecting with biological
systems and hence also medical applications, or with new applications for material properties and nonlinear fluids. The expected impact here will therefore add to
the well-proven and established record of statistical mechanics in general, while opening
new avenues in the following directions:
\begin{compactitem}
\item New materials and material properties: Search of bio- and meta-materials that exist
only in nonequilibrium conditions, or that must be built in a nonequilibrium environment.
\item Biomedical processes and transport: the cybernetics of active media and the steering
of active particles in biological (living) environments is expected to revolutionize medical
interventions and pharmacy via the targeted transport of drugs and new microinvasive
treatments and therapies.
\item Biomechanical structures and motility: Similar to the previous point, we need to
understand much better the structural and architectural aspects of living materials such as
the cytoskeleton of a cell or in general the mechanical behavior of tissue. That is
essential also for understanding the motility of cells and for the control of transduction
of signals to cell interior. Basic nonequilibrium physics complements the biochemical
and experimental research there.
\item The loss of coherence of quantum systems in contact with large thermal environments or
with many degrees of freedom limits our engineering abilities. We have obtained theoretical
proof that suitably driven nonequilibrium medium, with well-defined and controlled
properties, would allow the control and coherence of immersed (quantum) devices.
%
\item In industrial production, very fast or transient processes occur far from equilibrium
and parameters like stresses and temperatures are changing very rapidly.
\end{compactitem}

We see in all of the above items the ubiquitous role of the nonequilibrium paradigm. Its
transformational impact on science, technology and society are enormous and largely
unexploited. \TheProject promises a serious start toward these applications via
nonequilibrium physics.

Ideas from statistical mechanics and dynamical systems have in the past had huge impact on
other scientific fields that could in turn benefit from our advances.
%
In economy, microscopic models are used to understand large scale phenomena and a branch of
it is now called {\em econophysics}.
%
The fluxes of energy and matter for the weather, the climate and the environment are
strongly out of equilibrium. These fields of applications are of increasing importance given
the increase of industrial activity worldwide.

\subsection{Measures to maximise impact}

\paragraph{a) Dissemination and exploitation of results}

The major method of dissemination of our results is first the standard scientific practice
of open access publishing, and participation in international conferences and discussion
fora. The universities have important research and development centres where starting
spin-offs find resources and support.
%
There will be important contacts with these centres.

Dissemination of the results is done by the free web archive, in publications in the
standard specialized scientific journals (mostly non-commercial), by talks, by schooling and
in contacts in conferences. As members of the Editorial Boards of Journal of Physics A  (IOP), of the Journal of Statistical Physics (Springer), of Annales Henri Poincar\'e, of
Fundamental Theories of Physics (Springer) and of the liquid matter board of the European Physical Society, etc we can organize special issues and create
special volumes dedicated to aspects of the project. Indeed, as the project deals with some
very novel aspects that have not appeared in standard or even not so standard treatments of
irreversible thermodynamics, an important effort will be necessary and will be made to reach
also less specialized researchers as well as scientists involved in possible applications.
%
Obviously an important cost then also goes to traveling and participation in conferences and
international discussions. Again, since the project is not entirely mainstream and some
concepts/questions are quite new, the project needs to invest in visibility and in numerous
exchanges on international platforms.  All team members will be
expected to show great mobility.  During the years, the various members meet once at each
node with everybody, at a rate of roughly one meeting every 10 months. PIs stay a few days
for talks of everyone and discussions. Postdocs and students can stay
longer, extending their visits for a month. That will allow perfect exchange of ideas and
progress.  Similar considerations apply for inviting scientists where
costs include traveling, housing and subsistence.  The project will also invest (still under other
direct costs) in bringing together various points of view on the role of dynamical activity
and the construction of nonequilibrium statistical mechanics. For that reason we foresee to
organize two major conferences, one at around month 20, the other towards the end of the
project.  


\eucommentary{
  \begin{compactitem}
  \item Provide a plan for disseminating and exploiting the project results. The
    plan, which should be proportionate to the scale of the project, should
    contain measures to be implemented both during and after the project.
  \item Explain how the proposed measures will help to achieve the expected
    impact of the project.
  \item Where relevant, include information on how the participants will manage
    the research data generated and/or collected during the project, in
    particular addressing the following issues\footnote{For further guidance on
      research data management, please refer to the H2020 Online Manual on the
      Participant Portal.}:
    \begin{compactitem}
    \item What types of data will the project generate/collect?  o What
      standards will be used?
    \item How will this data be exploited and/or shared/made accessible for
      verification and re-use?  If data cannot be made available, explain why.
    \item How will this data be curated and preserved?
    \end{compactitem}
 %      
    You will need an appropriate consortium agreement to manage (among other
    things) the ownership and access to key knowledge (IPR, data etc.). Where
    relevant, these will allow you, collectively and individually, to pursue
    market opportunities arising from the project's results.\\
%
    The appropriate structure of the consortium to support exploitation is
    addressed in section~3.3.
%
  \item Outline the strategy for knowledge management and protection. Include
    measures to provide open access (free on-line access, such as the 'green' or
    'gold' model) to peer- reviewed scientific publications which might result
    from the project.%
    \footnote{Open access must be granted to all scientific publications
      resulting from Horizon 2020 actions. Further guidance on open access is
      available in the H2020 Online Manual on the Participant Portal.}\\
%
    Open access publishing (also called 'gold' open access) means that an
    article is immediately provided in open access mode by the scientific
    publisher. The associated costs are usually shifted away from readers, and
    instead (for example) to the university or research institute to which the
    researcher is affiliated, or to the funding agency supporting the research.\\
%
    Self-archiving (also called 'green' open access) means that the published
    article or the final peer-reviewed manuscript is archived by the researcher
    - or a representative - in an online repository before, after or alongside
    its publication.  Access to this article is often - but not necessarily -
    delayed ('embargo period'), as some scientific publishers may wish to recoup
    their investment by selling subscriptions and charging pay-per-download/view
    fees during an exclusivity period.
%
  \end{compactitem}
}

\paragraph{b) Communication activities}

\eucommentary{
  \begin{compactitem}
  \item Describe the proposed communication measures for promoting the project
    and its findings during the period of the grant. Measures should be
    proportionate to the scale of the project, with clear objectives. They
    should be tailored to the needs of various audiences, including groups
    beyond the project's own community. Where relevant, include measures for
    public/societal engagement on issues related to the project.
  \end{compactitem}
}

Various members of the team have larger networks of communication both towards the general
public (public outreach) as towards industrial players and groups of interest.  The first
aim there is to transfer knowledge and tools, and to help in optimizing public interest. An
important initiative will be the organization of ``physics meets industry days'' where each
time during a week, specific problems of industry or economic activity will be presented.
They will be treated by students and experts towards helping to solve these problems, with
direct feedback toward the industry of company.  Such initiatives exist already in some
countries but will be started up in other European countries, and with an additional
selection and expertise platform related to complex and nonequilibrium phenomena.



%%% Local Variables:
%%% mode: latex
%%% TeX-master: "proposal"
%%% End:


% ---------------------------------------------------------------------------
%  Section 3: Implementation
% ---------------------------------------------------------------------------

\section{Implementation}

\subsection{Project work plan}
\label{sec:wp}

\eucommentary{
Please provide the following:
\begin{compactitem}
\item brief presentation of the overall structure of the work plan;
\item timing of the different work packages and their components (Gantt chart or
  similar);
\item detailed work description, i.e.:
  \begin{compactitem}
  \item a description of each work package (table 3.1a);
  \item a list of work packages (table 3.1b);
  \item a list of major deliverables (table 3.1c);
  \end{compactitem}
\item graphical presentation of the components showing how they inter-relate (Pert
  chart or similar).
\end{compactitem}
%
Give full details. Base your account on the logical structure of the project and
the stages in which it is to be carried out. Include details of the resources to
be allocated to each work package. The number of work packages should be
proportionate to the scale and complexity of the project.\\
%
You should give enough detail in each work package to justify the proposed
resources to be allocated and also quantified information so that progress can
be monitored, including by the Commission.\\
%
You are advised to include a distinct work package on 'management' (see section
3.2) and to give due visibility in the work plan to 'dissemination and
exploitation' and 'communication activities', either with distinct tasks or
distinct work packages.\\
%
You will be required to include an updated (or confirmed) 'plan for the
dissemination and exploitation of results' in both the periodic and final
reports. (This does not apply to topics where a draft plan was not required.)
This should include a record of activities related to dissemination and
exploitation that have been undertaken and those still planned. A report of
completed and planned communication activities will also be required.\\
%
If your project is taking part in the Pilot on Open Research Data%
\footnote{%
  Certain actions under Horizon 2020 participate in the 'Pilot on Open Research
  Data in Horizon 2020'. All other actions can participate on a voluntary basis
  to this pilot. Further guidance is available in the H2020 Online Manual on the
  Participant Portal.},
you must include a 'data management plan' as a distinct deliverable within the
first 6 months of the project. A template for such a plan is given in the
guidelines on data management in the H2020 Online Manual. This deliverable will
evolve during the lifetime of the project in order to present the status of the
project's reflections on data management.\\
%
Definitions:
\begin{description}
\item[Work package] means a major sub-division of the proposed project.
\item[Deliverable] means a distinct output of the project, meaningful in terms of the project's overall
  objectives and constituted by a report, a document, a technical diagram, a software etc.
\item[Milestones] means control points in the project that help to chart progress. Milestones
    may correspond to the completion of a key deliverable, allowing the next phase of the
    work to begin. They may also be needed at intermediary points so that, if problems have
    arisen, corrective measures can be taken. A milestone may be a critical decision point in
    the project where, for example, the consortium must decide which of several technologies
    to adopt for further development.
\end{description}
%
Report on work progress is done primarily through the periodic and final reports. Deliverables
should complement these reports and should be kept to the minimum necessary.
}

\TheProject will begin by the setup of theoretical (\WPref{WPcompress},
\WPref{WPdissipation}) and experimental (\WPref{WPactive}, \WPref{WPbrown}) physical
models. At the same time, \WPref{WPcore} will lay the basis for the related theoretical
framework.
%
Then, building on the first round of results the work packages will feed on one another.

\subsection{Management and risk assessment}

\eucommentary{
  \begin{compactitem}
  \item Describe the organisational structure and the decision-making (including a list of
  milestones (table 3.2a)) .
\item Describe any critical risks, relating to project implementation, that the stated project's
  objectives may not be achieved. Detail any risk mitigation measures. Please provide a
  table with critical risks identified and mitigating actions (table 3.2b).
  \end{compactitem}
}

\subsection{Consortium as a whole}

\eucommentary{The individual members of the consortium are described in a
  separate section 4. There is no need to repeat that information here.\\
%
  \begin{compactitem}
  \item Describe the consortium. How will it match the project's objectives? How
    do the members complement one another (and cover the value chain, where
    appropriate)? In what way does each of them contribute to the project? How
    will they be able to work effectively together?
  \item If applicable, describe how the project benefits from any industrial/SME
    involvement.
  \item Other countries: If one or more of the participants requesting EU
    funding is based in a country that is not automatically eligible for such
    funding (entities from Member States of the EU, from Associated Countries
    and from one of the countries in the exhaustive list included in General
    Annex A of the work programme are automatically eligible for EU funding),
    explain why the participation of the entity in question is essential to
    carrying out the project.
  \end{compactitem}
}

\subsection{Resources to be committed}

\eucommentary{
  Please make sure the information in this section matches the costs as stated
  in the budget table in section 3 of the administrative proposal forms, and the
  number of person/months, shown in the detailed work package descriptions.\\
%
  \begin{compactitem}
  \item a table showing number of person/months required (table 3.4a)
  \item a table showing 'other direct costs' (table 3.4b) for participants where
    those costs exceed 15\% of the personnel costs (according to the budget
    table in section 3 of the administrative proposal forms)
  \end{compactitem}
}


\gantttaskchart[draft,xscale=.33,yscale=.33,milestones]

\subsubsection{Deliverables}\label{sec:deliverables}
\inputdelivs{9.3cm}

\subsubsection{Milestones}\label{sec:milestones}
\eucommentary{Milestones means control points in the project that help to chart progress. Milestones may
correspond to the completion of a key deliverable, allowing the next phase of the work to begin.
They may also be needed at intermediary points so that, if problems have arisen, corrective
measures can be taken. A milestone may be a critical decision point in the project where, for
example, the consortium must decide which of several technologies to adopt for further
development.}

\begin{milestones}
\milestone[id=start, month=1, verif={Public announcement of open positions.}]
%
{Starting up}
%
{Formal start of the project: Open the research positions, set up the website.}

\milestone[id=models,month=12, verif={}]
%
{Systems setup}
%
{Set up the experiments and the simulation software.}

\milestone[id=framework,month=24, verif={Availability of theoretical models. Publication of
  first technological report.}]
%
{Common theoretical framework}
%
{The physical properties of all model systems, theoretical and experimental, are formulated
  in a unified manner. The role of nonequilibrium in the properties of the systems is given
  a meaning.}

\milestone[id=data1,month=24, verif={Availability of experimental data.}]
%
{Experimental results}
%
{The first round of experiments provides validated data.}

\milestone[id=final,month=48, verif={Publication of final report and second
  technological report.}]%
{Final milestone}
%
{Join the theoretical model and experimental results in joint publications}

\end{milestones}

%%% Local Variables:
%%% mode: latex
%%% TeX-master: "proposal"
%%% End:



% ---------------------------------------------------------------------------
% Include Work package descriptions
% ---------------------------------------------------------------------------

\input{WorkPackages/WorkPackages}

\subsection{Management Structure and Procedures}
\input{management_structure_and_procedures.tex}

\draftpage
\subsection{Consortium as a Whole}

\eucommentary{\begin{compactitem}
\item
Describe the consortium. How will it match the project's objectives?
How do the members complement one another (and cover the value chain,
where appropriate)? In what way does each of them contribute to the
project? How will they be able to work effectively together?
\item
If applicable, describe the industrial/commercial involvement in the
project to ensure exploitation of the results and explain why this is
consistent with and will help to achieve the specific measures which
are proposed for exploitation of the results of the project (see section 2.3).
\item
Other countries: If one or more of the participants requesting EU funding
is based in a country that is not automatically eligible for such funding
(entities from Member States of the EU, from Associated Countries and
from one of the countries in the exhaustive list included in General
Annex A of the work programme are automatically eligible for EU funding),
 explain why the participation of the entity in question is essential to carrying out the project
\end{compactitem}
}

The consortium consists in a diversity of researchers, from the Czech republic, Germany, the
Netherlands, Italy and Belgium, all much engaged in international collaborations and
organizations.
%
The ages range between 55 and 35 years for the principal investigators with a large backbone
of researchers and infrastructure, allowing easy intake of students and postdocs. An equal
opportunity policy is maintained, with special emphasis on including excellent women
researchers in the team. For example, recent collaborations of the coordinator included also
supervision of and joint work with Soghra Safaverdi (woman from Teheran), with Simi Thomas
(woman student from India) and with Urna Basu (woman postdoc from India).  Members of the
consortium know each other from sharing the same objectives and from complimentary
expertise.
%
Applied mathematics, mathematical physics, theoretical physics, soft condensed matter and
liquid matter labs join here in a global effort around nonequilibrium physics, with the
special aim of developing knowledge and tools for control and manipulation of nonequilibrium
systems. Some members already work together, and have established research connections,
e.g. Leuven-Prague, Stuttgart-Leipzig and Leuven-Padova.  Other collaborations are more
recent, such as Leuven-Stuttgart and Padova-Leipzig.  Still other visits and discussions
started since about one year, Leuven-Eindhoven en Leuven-Leipzig.
%
There will continue to be many exchanges of students and postdocs and mutual visits to bring
the results forward.  In particular, important exchanges between the theoretical institutes
and experimental labs are foreseen.  Many of these members have contacts with research
institutes that allow easy access to industrial and commercial involvement.  For example,
the coordinator has collaborations and supervises students at imec (Leuven), one of the main
players in technological innovation in Europe. From November 2015 we will enter in direct
contact with the managers at imec for discussing new avenues in quantum technology, with the
damping of decoherence through contacts with nonequilibria as one of the new (and not te be
disclosed) possibilities for a major breakthrough.

%%% Local Variables:
%%% mode: latex
%%% TeX-master: "proposal"
%%% End:


\draftpage

\subsection{Resources to be Committed}
\input{resources}

% ---------------------------------------------------------------------------
%  Section 4: Members of the Consortium
% ---------------------------------------------------------------------------

\newpage

\eucommentary{This section is not covered by the page limit.\\
The information provided here will be used to judge the operational capacity.}

\section{Members of the Consortium}

\subsection{Participants}

\eucommentary{Please provide, for each participant, the following (if available):\\
\begin{compactitem}
\item
a description of the legal entity and its main tasks,
with an explanation of how its profile matches the tasks in the proposal;
\item
a curriculum vitae or description of the profile of the persons,
including their gender, who will be primarily responsible for carrying
out the proposed research and/or innovation activities;
%
this includes a description of the profile of the to-be-recruited personnel
\item
a list of up to 5 relevant publications, and/or products, services
(including widely-used datasets or software), or other achievements
relevant to the call content;
\item
a list of up to 5 relevant previous projects or activities, connected
to the subject of this proposal;
\item
a description of any significant infrastructure and/or any major items
of technical equipment, relevant to the proposed work;
\item
any other supporting documents specified in the work programme for this call.
\end{compactitem}}

\begin{sitedescription}{KUL} \label{desc:KUL}

KU Leuven is currently by far the largest university in Belgium in terms of
research funding and expenditure (EUR 426.5 million in 2014), and is a charter
member of LERU. KU Leuven conducts fundamental and applied research in all
academic disciplines with a clear international orientation.  Leuven
participates in over 540 highly competitive European research projects (FP7,
2007-2013), ranking sixth in the league of HES institutions participating in
FP7. In Horizon 2020, KU Leuven currently has been approved 79 projects.

KU Leuven takes up the 9th place of European institutions hosting ERC grants (as
first legal signatories of the grant agreement). To date, the
\href{http://www.kuleuven.be/english/research/EU/p/erc}{78 ERC Grantees}
(including affiliates with VIB and IMEC) in our midst confirm that KU Leuven is
a breeding ground (51 Starting Grants) and attractive destination for the
world's best researchers. The success in the FP7 and Horizon 2020 Marie
Sklodowska Curie Actions is a manifestation of the three pillars of KU Leuven:
research, education and service to society. In our
\href{http://www.kuleuven.be/english/research/EU/p/horizon2020/es/msca}{170
Actions}, of which 76 Initial/European Training Networks, hundreds of young
researchers have been trained through research and have acquired the necessary
skills to transfer their knowledge into the world outside academia.
%
KU Leuven Research \& Development (LRD) is the technology
transfer office (TTO) of the KU Leuven. Since 1972 a multidisciplinary team of
experts guides researchers in their interaction with industry and society, and
the valorisation of their research results (101 spin offs, \dots).

Within the KU Leuven, the Institute for Theoretical Physics (ITF) has 8 permanent staff
members and about 15 PhD students and 10 postdocs in three different areas of Modern
Theoretical Physics: High-Energy Physics, Mathematical Physics and Statistical Physics.
%
The support staff of the ITF (one secretary and one IT support person) ensures excellent
working conditions and, in combination with the administration of the KU Leuven, provides an
ideal environment for developing ambitious research projects.
%
The Department of Physics is host to many excellent researchers and six of its professors
are currently running a ERC grant.

The ITF enjoys regular contacts with the nearby IMEC, a research institute dedicated to
nanoelectronics. Prof. Christian Maes supervises a student jointly with IMEC.

\subsubsection*{Curriculum vitae of the investigators}

\begin{participant}[type=PI,PM=12,gender=male,salary=5500]{Christian Maes}
\url{http://fys.kuleuven.be/itf/staff/christ}

Full professor at the KU Leuven and director of the Institute for Theoretical Physics.

Christian Maes is a leading scientist in the field of mathematical and nonequilibrium statistical mechanics, regularly
invited as a keynote to scientific events worldwide.
%
He has published 150 articles in peer-reviewed journals,
is currently an associate editor or member of the editorial board of 4 leading international journals,
has supervised 14 PhD theses (2 more ongoing) and 11 postdoctoral researchers,
is expert and reviewer for many scientific institutions and
is a member of various international evaluation commissions.

\end{participant}

%%% Local Variables:
%%% mode: latex
%%% TeX-master: "../proposal"
%%% End:

\begin{participant}[type=R,PM=12,gender=male,salary=5500]{Pierre de Buyl}

Postdoctoral researcher at the KU Leuven. \url{http://pdebuyl.be/}

Pierre de Buyl holds a PhD in Physics (2010) and is specialized in theoretical and
computational studies in statistical physics. He has published 15 papers in international
peer-reviewed journals and also made conference contributions (posters, articles in
conference proceedings and two invited talks). He has worked at the Université libre de
Bruxelles, the University of Toronto and is now at the KUL.
%
His early work is about the consequences of long-ranged interactions on the dynamical
evolution of models relating to plasmas and gravitation.
%
He contributed innovative insight through the direct resolution of the Vlasov equation.
This work provides an alternative to particle-based simulations and a unique point of view
on timely research questions. The resulting simulation code is available under an
open-source license and is the subject of a dedicated publication.
%
His current focus is on so-called nanomotors, a class of devices that transforms a fuel into
motion for the motor (hence the name). The operating conditions are fluctuating and strongly
out of equilibrium, in direct relation with \TheProject.

In addition to his research skills, de Buyl also participates in the organization of
scientific events (web site and database for the {\em European Conference on Complex Systems
  2012}, organizer of the conferences {\em EuroSciPy 2012} and {\em EuroSciPy 2013},
webmaster for EuroSciPy since 2013 and proceedings editor in 2013 and 2014).

de Buyl started his career as a teaching assistant and has thought for several hundred hours
already, from first year physics classes to doctoral training. He has supervised students
for research projects from the third year of Bachelor and for Master Thesis work (one at the
Université libre de Bruxelles in 2013 and one in the coming year at the KU Leuven).

\end{participant}


\begin{participant}[type=res,PM=48,salary=5500]{NN}
A postdoc will be hired to work on the project. We aim to hire someone who has a strong
background in theoretical statistical mechanics and a diverse research experience
(applications, computations, etc).
%
This person will be in contact with several other nodes and should be willing to engage in a
large collaborative effort.
\end{participant}

\begin{participant}[type=res,PM=48,salary=3500]{NN}
A PhD student will be hired for the duration of the project that matches the duration of a
PhD thesis in Belgium. The ITF graduates students with an advanced knowledge of statistical
mechanics, ensuring that skilled candidates will exist for the position.
%
The opening will indeed be international and open to competent candidates from any
origin.
\end{participant}

\begin{participant}[type=res,PM=24,salary=3932]{NN}
We will hire an experienced part time project manager to help with the overall management
during the whole duration of \TheProject.
\end{participant}

\subsubsection*{Publications, achievements}

\begin{compactenum}
\item M. Baiesi, C. Maes and B. Wynants {\em Fluctuations and response of nonequilibrium
  states}, Physical Review Letters {\bf 103}, 010602 (2009). This article is already cited
137 times according to Google Scholar, an achievement for a theoretical work in statistical
physics. This article highlights the increasing interest in nonequilibrium physics.
\item {\tt vmf90} software for the numerical resolution of the Vlasov equation:
\url{https://github.com/pdebuyl/vmf90}. This demonstrates the applicant's ability to write
open-source software of scientific relevance.
\end{compactenum}


\subsubsection*{Previous projects or activities}

\begin{compactenum}
\item Organization of the international school ``Fundamental problems in Statistical
Physics''.
%
This school is organized every four years since the 1970s (since 2005 in Leuven) and brings
together the world's most influential experts on Statistical Physics.
\item Membership of expert committees for the Irisch Research Council, the ERC Starting
Grants in Mathematics, of the steering committee of the European Science Foundation
Programme Random Geometry of Large Interacting Systems and Statistical Physics (RGLIS),
among others.
\item Partner in national collaborative projects (Belgian federal government).
\end{compactenum}

\end{sitedescription}



\begin{draft}
\vspace{1cm}\TOWRITE{PAR1P1}{Complete check list below -- delete completed items if you wish}

\begin{verbatim}
- [ ] checked that sum of person months put into finance request is
  the same as sum of person months associated with the Work Packages
  (in proposal.tex, as defined as part of the \begin{workpackage}"
  command.
  
- [ ] completed site specific resource summary in resources.tex,
  including table of non-staff costs.

\end{verbatim}
\end{draft}

%%% Local Variables: 
%%% mode: latex
%%% TeX-master: "../proposal"
%%% End: 

\begin{sitedescription}{TUE}

TUE is an internationally leading research university specialised in engineering science and technology. TUE is a constant presence in the top of various rankings, all acknowledging in particular the research carried out in collaboration with the industry, and currently manages more than 150 EU-funded projects. TUE offers 50 different educational programs to about 9000 students of various levels, and employs more than 3000 academic and administrative personnel.

The TUE research team in this proposal is part of both the Department of Mathematics and Computer Science and the Institute for Complex Molecular Systems. 
The Department Mathematics and Computer Science (MCS) unites all activities on campus in the classical scientific disciplines of mathematics and computer science. Both research and education have recently been evaluated as excellent by international committees. 40\% of the employees at MCS are of non-Dutch origin, thus creating a truly multi-cultural environment. 

The Institute for Complex Molecular Systems (ICMS) is a hotbed for interdisciplinary interaction in research and education across the university. Its mission is to become the leading international multidisciplinary institute for research and education in the area of the engineering of complex molecular systems.

The joint affiliation with both the Department of Mathematics and Computer Sciences and the ICMS provides the TUE research team with both excellent mathematical infrastructure and broad experimental embedding.


\subsubsection*{Curriculum vitae}

% Curriculum of the personnel at this institution. This includes
% to-be-hired people for which there is a tentative candidate.

\input{CVs/Mark.Peletier.tex}

\input{CVs/Adrian.Muntean.tex}
% For other to-be-hired person, please include here something like:
% \begin{participant}[type=res,PM=3,salary=5900]{NN}
%  <a _short_ description of the qualifications of whom you want to hire>
% \end{participant}

\begin{participant}[type=res,PM=48]
A PhD student will be hired for the duration of the project that matches the duration of a
PhD thesis in the Netherlands. 
%
The opening will be international and open to competent candidates from any
origin.
\end{participant}


\subsubsection*{Publications, products, achievements}

\begin{compactenum}
\item {In a series of papers, starting with \emph{Adams et al., Communications in Mathematical Physics, 307:791 (2011)}, Peletier and co-authors identified and explored the deep relations between gradient flows on one hand and large deviations of stochastic processes on the other. These relations create new understanding of the mathematical structures describing physical phenomena at the mesoscale, and point the way towards understanding the strongly nonequilbrium systems of this proposal.}
\item {M. A. Peletier, G. Savar\'e, and M. Veneroni, From diffusion to reaction via $\Gamma$-convergence, SIAM Journal on Mathematical Analysis, 42(4), pp. 1805--1825, 2010.
This paper has sparked renewed interest in the analysis community in the old problem of a stochastic particle escaping from a potential well, by introducing a new method to combine reactive and diffusive effects. It was selected as a SIGEST paper in SIAM Review 54(2) in 2012.}
\item {Peletier is a co-founder and board member of the ICMS, and both Peletier and Muntean are prominent members of this interdisciplinary institute.}
\end{compactenum}

\subsubsection*{Previous projects or activities}

\begin{compactenum}
\item {Peletier is and has been PI on many national and international projects, including an EU Seventh-framework ITN project `Fronts and Interfaces in Science and Technology'.}
\item {Peletier and Muntean have together and separately organized over 50 meetings, workshops, and conferences.}
\end{compactenum}

\subsubsection*{Significant infrastructure}

{Both CASA and the ICMS have significant in-house computing resources (several computing clusters) as well as access to national computing facilities at SARA (Amsterdam).}
\end{sitedescription}

%%% Local Variables:
%%% mode: latex
%%% TeX-master: "../proposal"
%%% End:

\begin{sitedescription}{ULEI} \label{desc:ULEI}

 Universität Leipzig was founded in 1409 and is the second oldest university in Germany where teaching has continued without interruption. 
  Today it offers a wide spectrum of academic disciplines at 14 faculties with more than 150 institutes. 
  It is a member of the German U15, a strategic alliance of 15 major German research universities. 
  The node participants are located at the Institute for Experimental Physics I and at the Institute for Theoretical Physics, respectively, which are parts of the Faculty of Physics and Earth Sciences. 
  The faculty also comprises institutes for meteorology, geology, geophysics and geography. 
  It counts among the leading faculties in terms of research output and external funding, within the university, and hosts several ERC grantees. 
  It is also the first faculty of the university that recently got its research activities evaluated by an external board of international experts. 
  The physics institutes make major contributions to several collaborative research centers (SFBs) funded by the German Science Foundation (DFG) and to the interdisciplinary Graduate School ``Leipzig School of Natural Sciences -- Building with Molecules and Nano-objects'' (www.buildmona.de), founded by a grant from the German Excellence Initiative, which has so far enrolled close to 200 PhD candidates. 
  They maintain collaborations and joint grants with a large number of independent international and local research institutes for fundamental and applied science and industrial partners. 

\subsubsection*{Curriculum vitae of the investigators}

\begin{participant}[type=R,PM=12,gender=male,salary=5500]{Klaus Kroy}

Professor at the Institute of Theoretical Physics, Universität Leipzig.

Klaus Kroy is a theoretical physicist and an expert in the field of
soft mesoscopics (nonequilibrium dynamics of colloids and polymers; active particles; cytoskeleton and tissue mechanics; single-molecule force spectroscopy; aeolian sand transport and structure formation)

He has published about 60 articles in peer-reviewed journals (also in
Nature Physics, Nature Communications, PNAS, PRL) 

He has in the past supervised 3 postdocs, 5 PhD students, and 19
master students 

He is a Member of the German Physical Society, of the International Max--Planck Research Group
Mathematics in the Sciences (Leipzig), and he recently received grants from
the German Excellence Initiative (Graduate School ``BuildMoNa''),
the DFG-Forschergruppe FOR 877, the German
priority programm SPP1726 (DFG), the German Israel
Foundation, the ESF, and a DFG-individual-grant.


\end{participant}

\begin{participant}[type=R,PM=12,gender=male,salary=5500]{Frank Cichos}

Professor at the Institute of Experimental Physics I, Universität Leipzig. \url{http://www.uni-leipzig.de/~mona}

Frank Cichos is an experimental physicist and an expert optical microscopy and optical single molecule detection (photothermal single molecule detection; active particles; single molecule trapping; single molecule dynamics in soft matter)

He has published about 76 articles in peer-reviewed journals (also in
Nano Letters, ACS nano and PRL) 

He has in the past supervised 2 postdocs, 15 PhD students, and about 20
master students.

He is a Member of the German Physical Society and the American Physical Society. He recently received grants from the German Excellence Initiative (Graduate School ``BuildMoNa''), the DFG-Forschergruppe FOR 877, the German priority program SPP1726 (DFG), the DFG Sonderforschungsbereich TRR102 and a joint  DFG-ANR-individual-grant. He is the co-speaker of the DFG Sonderforschungsbereich TRR102 and has been the speaker of the DFG-Forschergruppe FOR 877.

\end{participant}


\begin{participant}[type=res,PM=48,salary=5500]{NN}
A postdoc will be hired to work on the project. We aim to hire someone who has a strong
background in theoretical statistical mechanics and a diverse research experience. The person will keep contact to the other project nodes and the experimental project partners.
\end{participant}

\begin{participant}[type=res,PM=36,salary=5500]{NN}
A postdoc researcher is required to carry out the experiments on hot Brownian motion at nanosecond timescales. The postdoc should have a solid background in optical tweezer experiments and fast positional detection. He shall work closely with the theoretical project partners and keep close contact to the other nodes.
\end{participant}

\subsubsection*{Publications, achievements}

\begin{compactenum}
\item M. Gralka, K. Kroy, Inelastic mechanics: A unifying principle in biomechanics. 
Biochimica et Biophysica Acta (BBA)--Molecular Cell Research 2015

\item S. Schöbl, S. Sturm, W. Janke, K. Kroy, Persistence-Length Renormalization of Polymers in a Crowded Environment of Hard Disks.
Physical Review Letters {\bf 113} 238302 (2014).

\item J. T. Bullerjahn, S. Sturm, K. Kroy, Theory of rapid force spectroscopy. Nature Communications {\bf 5} 4463 (2014).

\item O. Otto, S. Sturm, N. Laohakunakorn, U. F. Keyser, K. Kroy, Rapid internal contraction boosts DNA friction.
Nature Communications {\bf 4} 1780 (2013).

\item D. Chakraborty, M. V. Gnann, D. Rings, J. Glaser, F. Otto, F. Cichos, K. Kroy, 
Generalised Einstein relation for hot Brownian motion. EPL (Europhysics Letters) {\bf 96} 60009 (2011).

\item D. Rings, R. Schachoff, M. Selmke, F. Cichos, K. Kroy. 
Hot Brownian Motion.  Physical Review Letters {\bf 105},  090604 (2010). According to Google Scholar, this article has been cited 81 times. It highlights the interest in fundamental non-equilibrium fluctuation dissipation theorems and its experimental verification.

\item M. Braun, A. Bregulla, K. Günther, N. Mertig, F. Cichos Single Molecules Trapped by Dynamic Inhomogeneous Temperature Fields Nano Lett. 15 5499 (2015).

\item A. Bregulla, H. Yang, F. Cichos Stochastic Localization of Micro-Swimmers by Photon Nudging ACS Nano 8 6542 (2014).

\item M. Selmke, M. Braun, F. Cichos Photothermal Single Particle Microscopy: Detection of a Nanolens ACS Nano 6 2714 (2012).

\end{compactenum}

\subsubsection*{Previous projects or activities}

\begin{compactenum}
\item Project lead (speaker) of the research unit 877 ``From Local Constraints to Macroscopic Transport'' of the German Science Foundation (DFG).
%
\item Organization of the "Hot Nanostructures" workshop series.
This workshop has been initiated in 2011 by the two principle investigators in Leipzig and is continued every two years since then. It is focused especially on problems of non-equilibrium physics highlighting theory and experiments at inhomogeneous temperatures. The workshop will return in 2017 to Leipzig.

\item Partner in international collaborative projects (Agence National de la Recherche, ANR - German Science Foundation, DFG)
\end{compactenum}

\subsubsection*{Significant infrastructure}

\end{sitedescription}

\begin{draft}
\vspace{1cm}\TOWRITE{PAR1P1}{Complete check list below -- delete completed items if you wish}

\begin{verbatim}
- [ ] checked that sum of person months put into finance request is
  the same as sum of person months associated with the Work Packages
  (in proposal.tex, as defined as part of the \begin{workpackage}"
  command.
  
- [ ] completed site specific resource summary in resources.tex,
  including table of non-staff costs.

\end{verbatim}
\end{draft}

%%% Local Variables: 
%%% mode: latex
%%% TeX-master: "../proposal"
%%% End: 

\input{Participants/Padova.tex}
\input{Participants/FZU.tex}
\begin{sitedescription}{USTUTT} \label{desc:USTUTT}

{\bf Universität Stuttgart -- A Research University of International Standing:}\\
The Universität Stuttgart lies right in the centre of the largest high-tech region of Europe. We are surrounded by a number of renowned research facilities and have such global players as Daimler or IBM as our neighbours. We were founded in 1829 and over the years this technical institution has developed to the research intensive university that it is today. Our main emphasis is on engineering and the natural sciences.
%
Indicators of our excellent status are the two projects that were successful in the recent {\it Excellence Initiative} sponsored by both the Federal and the State governments. One project is the Cluster of Excellence {\it Simulation Technology} and the other, the Graduate School {\it Advanced Manufacturing Engineering}.

{\bf Experience with EU research funding:}\\
The University of Stuttgart has extensive experience with the various funding programs of the European Commission and has been the {\it leading German university in FP6}, both in number of projects (184) and in terms of funding (54 Mio. \euro). In FP7, it was yet again {\it among the most successful German universities} with 246 projects funded and a total budget of 94 Mio. \euro. The University has consistently been involved in Marie-Curie-projects in previous framework programs. In FP7, it participated in 13 Marie Curie actions.

\subsubsection*{Curriculum vitae of the investigators}

\begin{participant}[type=R,PM=12,gender=male,salary=5500]{Matthias Krüger}

Matthias Krüger joined the University of Stuttgart in October 2012 and is leading the
independent research group {\it Non-equilibrium Systems} located at the Max Planck Institute
for Intelligent Systems, Stuttgart. Before coming to Stuttgart, he was a postdoc at the
Massachusetts Institute of Technology in Cambridge, USA.  In 2009, he gained his Doctoral
degree as a theoretical physicist at the University of Konstanz, Germany.

He theoretically studies different aspects of nonequilibrium statistical physics, including
fluids far from equilibrium, fluctuation (Casimir) forces under nonequilibrium conditions,
as well as radiative energy transport on the nano-scale.
%
He recently received an Emmy Noether Grant from the German Research Foundation (DFG), a
prestigious program that allows young researchers early independence.  Before, he was
supported through other programs of DFG and Fulbright and is a regular committee member of
the German National Academic Foundation.

He published around 30 articles in peer-reviewed journals, and, including ongoing projects, supervised 2 postdocs, 1 Phd student and 3 undergraduate students.

\end{participant}
\input{CVs/Clemens.Bechinger.tex}

\begin{participant}[type=res,PM=48,salary=5500]{NN}
Postdoctoral Researcher.
\end{participant}
\begin{participant}[type=res,PM=36,salary=5500]{NN}
PhD Student.
\end{participant}

\subsubsection*{Publications, achievements}

\begin{compactenum}
\item Leadership.
\item Coauthoring.
\end{compactenum}

\subsubsection*{Previous projects or activities}

\begin{compactenum}
\item Organization.
\item Partner.
\end{compactenum}

\subsubsection*{Significant infrastructure}

\end{sitedescription}


\subsection{Third Parties Involved in the Project (including use of third party resources)}
\label{section:ThirdParties}

\paragraph{Third Party 1}\ 

\eucommentary{Please complete, for each participant, the table
(see page 27 of "VRETemplate.PDF"),
or simply state "No third parties involved", if applicable.}

Third Party 1 (hereafter TP1) will work on the project.

\paragraph{Other participants}\ 

For other participants, the only subcontracting costs will be for audit.

\bgroup
\def\arraystretch{1.5}  % 1 is the default
\noindent \begin{tabular}{|p{0.6\textwidth}|c|}
\hline
Does the participant plan to subcontract certain
tasks & Yes \\
\hline
\multicolumn{2}{|l|}{Audit} \\
\hline
Does the participant envisage that part of its work
is performed by linked third parties & No \\
\hline
\multicolumn{2}{|l|}{} \\
\hline
Does the participant envisage the use of
contributions in kind provided by
third parties & No \\
\hline
\multicolumn{2}{|l|}{} \\
\hline
\end{tabular}
\egroup

%No third parties involved.

% ---------------------------------------------------------------------------
%  Section 5: Ethics and Security
% ---------------------------------------------------------------------------

\newpage

\section{Ethics and Security}

\eucommentary{This section is not covered by the page limit.}

\subsection{Ethics}

\eucommentary{
If you have entered any ethics issues in the ethical issue table in the administrative proposal forms, you must:\\
$\bullet$ submit an ethics self-assessment, which: \\
-- describes how the proposal meets the national legal and ethical requirements of the
country or countries where the tasks raising ethical issues are to be carried out;\\
-- explains in detail how you intend to address the issues in the ethical issues table, in
particular as regards:
research objectives (e.g. study of vulnerable populations, dual use, etc.),
research methodology (e.g. clinical trials, involvement of children and related
consent procedures, protection of any data collected, etc.),
the potential impact of the research (e.g. dual use issues, environmental damage,
stigmatisation of particular social groups, political or financial retaliation,
benefit-sharing, malevolent use , etc.)\\
$\bullet$ provide the documents that you need under national law (if you already have them), e.g.:\\
-- an ethics committee opinion;\\
-- the document notifying activities raising ethical issues or authorizing such activities\\
If these documents are not in English, you must also submit an English summary of them
(containing, if available, the conclusions of the committee or authority concerned).\\
If you plan to request these documents specifically for the project
you are proposing, your request must contain an explicit reference to the project title}

\subsection{Security}

Please indicate if your proposal will involve:

\begin{compactitem}
\item activities or results raising security issues: NO
\item 'EU-classified information' as background or results: NO
\end{compactitem}
\end{proposal}
\TOWRITE{ALL}{Search through final.pdf ('make final') and look for questions marks ?? and XX and YY and XYZ as place holders where people intended to later add a link, or where a link is broken.}
\end{document}

%%% Local Variables:
%%% mode: latex
%%% TeX-master: t
%%% End:

